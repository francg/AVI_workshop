\subsection{Smart city learning}
The concept of smart-city has also been used in many different context and is associated with distinctive and innovative aspects that are often quite different. Big diversities are observed on the reasons \textit{why} different cities are defined as \textit{smart}.

This situation is the consequence of the lack of a clear and recognized definition of smart city.

Komninos\cite{komninos_intelligent_2002}, in his attempt to delineate the intelligent city, (perhaps the concept most closely related to the smart city), sees intelligent (smart) cities as ``\textit{territories with high capacity for learning and innovation, which is built\textendash in the creativity of their population, their institutions of knowledge creation, and their digital infrastructure for communication and knowledge management}''.

Smart cities are also a powerful ecosystem for learning. Smart city learning aim to support the improvement of all key factors contributing to the regional competitiveness: mobility, environment, people, quality of life and governance. The approach is aimed at optimizing resource consumption and saving time improving flows of people, goods and data\footnote{http://www.mifav.uniroma2.it/inevent/events/sclo/}.

Education in this context is pursued as a bottom-up process, where person and places are central. Smartness from a learning perspective exists both in the ambient data collected and among the communities that exists within a city.

The separation between student and teacher will fade out. Their role will be content or situation dependent: everybody will be a learner and the relation between persons will get a bigger role.


\subsection{Characteristics of Smart City Learning Applications}

%%...what are smart city learning applications?
%%...what kind of learning and what learning goals?

Technologies like mobile devices, tags, web based applications, geographical information and e-learning systems have already been used to develop smart city learning applications on the field\cite{perez-sanagustin_multichannel_2013}\cite{delfatto_geographic_2013}.

%%%%%%
Smart city learning applications consist in the implementation of urban informatics techniques and approaches to promote innovative engagement strategies\cite{amayocaldwell_urban_2013}.
Studies found that urban informatics provide an innovative opportunity to enrich students' place of learning within the city\cite{amayocaldwell_urban_2013}.

No doubt that among the consequences of such attention there is an acceleration in supporting the integration and embedding of ICT within physical environments to realize what has been defined the \textit{everyware}\cite{giffinger_smart_2007}.

% Other studies focused on the transformation from a \textit{space} of learning to a \textit{place} of learning in the smart city context, this is achieved through the exposure of real world urban sites and issues\cite{amayocaldwell_urban_2013}.

Some critical factors in smart city learning scenarios can be addressed with the help of TUIs. We propose a list of overlapping areas between the strengths of TUIs and the challenges of SCL scenarios.

The following aspects of TUIs can support and empower SCL applications.

\subsubsection{Data collection}
Sensor data can be captured embedding proper electronics in augmented objects and public or private spaces.
Some of this \textit{sensing} capabilities (e.g. accelerometers, gyro) are necessary to actually allow sensor-based user interactions; for example detecting the presence of public in a space and react with an appropriate feedback. Besides this, raw data sensed can be manipulated and processed also for other purposes more connected to the application logic rather than remaining limited to only sense user interaction.
The opportunity to flexibly and quickly prototype custom devices offered by todays' toolkits such as Arduino, allow to quickly explore multiple design choices adopting several different sensors.
The opportunity to situate sensing capabilities both in the environment and physically on the users widens the domain where data collection take place.

% Mobility of sensors embedded in TUIs vs city infrastructure

\subsubsection{Data visualization}
Citizens living the city can be subjected to common life situations where the best interfaces for accessing information can be fairly different. For example train schedule information can be provided by a screen-based interface when at home or by a wearable interface while riding a bike to the station. The capability of delivering the right content at the right time is important because it could improve learning efficacy. Tangible interfaces can deliver information using non distracting interfaces such as haptic or ambient feedbacks that might be more appropriate than screens when on the go. That can shuttle simple information to the user in a very fast and effective way.
Simple output components can be combined to implement custom-designed and tailored output primitives. For example a custom shaped led matrix.

\subsubsection{Data processing} %How data processing is used for SCL applications?
Processing of sensed data can introduce a variety of challenges and demands for compromises. Being TUIs connected devices by nature, two main alternatives are feasible, depending on the nature of data and the processing power available. Data can be either processed on-board, transmitted to an external service for storing and/or processing. An hybrid approach can also be used, where data is pre-elaborated or filtered on board and then transmitted for further processing.

\subsubsection{User interaction} %this overlaps with sec 2.1.2
In many smart city scenarios, users cannot always devote their full attention to interact with devices. TUIs helps at this regard allowing several interaction strategies that can be less distracting and immersive when the scenario does not require it.
Background and foreground interaction methods are both implementable, as well as multi-modal strategies where users can interact via different channels at the same time. Interaction can also be distributed in the space, involving actions performed on/with different augmented objects situated in the space.

\subsubsection{Sharing of artifacts and information}
Technologies like big screens and mobile devices, often used in smart city learning scenarios, present some constraints that are quite rigid and difficult to overcome. For example mobile devices are proven to support cooperative activities and information sharing, but they physically remain strictly personal apparatus that are not comfortably shared even among friends and relatives.
Public displays are instead useful only to present information that is relevant for a defined geographical place, usually where the display itself is located. The nature of information should also be generic enough to not introduce privacy concerns.
With TUIs it is possible to fill the gap between public and personal/private. Devices can be tailored and adapted to support use cases where the users are comfortably sharing objects and information.
The added flexibility allow for several level of experience that can freely space between the \textit{private} and the \textit{public}.

\subsubsection{Learning, smart cities and ubiquitous}
Tools, as well as ICT applications and the Internet itself, may facilitate higher order skills and thus mediate learning, in schools and beyond\cite{kashtan_outdoors_2013}. Indeed, Engestrom\cite{engestrom_non_1991} argues against the separation between schooling and other learning experiences; he refers to it as the encapsulation of school learning. To overcome this encapsulation, he offers emphasis on the role of mediating artifacts in human cognition and learning.
This project achieves overcoming the encapsulation of school learning by using ICT tools to mediate the environment to the students, as well as to expose the products to the community\cite{kashtan_outdoors_2013}.

In \cite{luff_mobility_1998} an analysis of technology-supported coordination mechanisms is provided for several scenarios.
After investigating a urban scenario in London's underground they reported that:

\textit{For example, in distributing technologies around the environment developing support for station supervisors would appear to be a prototypical case of "ubiquitous computing" or
"augmented reality". However, although typical developments in these areas aim to support tasks and activities by augmenting everyday artifacts with computational capabilities it is not all that clear which artifacts are most relevant for such enhancement or what capabilities should be augmented.}

It is also mentioned that this type of technologies are able to support coordination mechanisms demonstrated to be essential for the scenarios examined, like micro-mobility, ubiquitous information retrieval and glances:

\textit{Nevertheless, it appears from the studies considered in this paper that the micro-mobility of objects may be critical when considering how to support co-present, collaborative activities. To provide for this may require not only both mobile and fixed devices, but quite novel support for mobility that focuses on the moment-to-moment manipulation of objects.}

