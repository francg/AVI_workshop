Pervasive information visualization and interaction are fundamental tools to support learning in smart cities, for example to promote sustainable behaviors and social interaction. In this setting, information supporting learning goals is usually provided via screen-based interfaces such as public large displays and smart-phones. Those interfaces pose several usability issues and ignore the space of opportunities offered by research in tangible and embodied interaction.
Building on a review of existing work, we identify the main limitations of traditional approaches based on large displays and smart-phones apps, first from a technological point of view then connecting to implications for design, user interaction and experience.

In the paper we reflect on four implications related to the use of Smart Objects (SO) for smart city learning (SCL) applications: (i) Sensor data collection in the urban environment; (ii) Rich interaction techniques between users and technology; (iii) Background interaction and glanceable events in addition to foreground interaction for information display; (iv) Multimodal interaction, observable through different endpoints or objects distributed in the space.

We focus more in detail on new design opportunities, developing possible scenarios of interest that involve smart city learning.
