Pervasive information visualization and interaction are fundamental tools to support learning in smart cities (SCL), for example to promote sustainable behaviours and social interaction. 

Building on a review of existing work, we identify the main limitations of traditional approaches based on large displays and smart-phones apps, first from a technological point of view then connecting to implications for design, user interaction and experience.

In the paper we propose a set of authoring primitives at different semantic levels, ranging from more generic internet of things (IoT) primitives to more domain specific approaches connected to learning in SCL applications.

We focus more in detail on new design opportunities, developing possible scenarios of interest that involve SCL. The use of a toolkit for rapid prototyping is proposed as a valuable support instrument for application design and development.
