\subsection{Characteristics of Tangible Interfaces and Smart Objects}
Tangible user interfaces (also called graspables, tokens, phicons, containers) denote systems that rely on ``tangible manipulation, physical representation of data and embeddedness in the real space'', allowing for an embodied interaction with digital information. Embodied interaction, as defined by Dourish \cite{dourish_where_2004}, is a collection of trends emerged in HCI, relying on the common ground to provide a more natural user interaction with digital information.

Embodied interaction takes the interaction ``off the screen'' to the real world by distributing inputs and outputs in space rather than in time, desequentialising interaction and reducing the gap between where the information is created and where it is accessed. In this picture TUIs seamlessly integrate both representation and control of computation into physical artifacts: ``By treating the body of the device as part of the user interface -an embodied user interface- we can go beyond the manipulation of a GUI and allow the user to really directly manipulate an integrated physical-virtual device''.
When these artifacts also resemble and retain the functionalities of traditional objects, they can be called smart or augmented objects.

TUIs and smart objects (SOs) allow interaction designers to be free to experiment with new types of metaphors, taking advantages of the users' physical skills and providing interfaces which exploit people's knowledge with the everyday non-digital world.

Unlike traditional ICT systems, the interactional paradigm is not one device per person. Rather, a single person interacts with a collection of devices that are orchestrated to expose coherent behaviours.