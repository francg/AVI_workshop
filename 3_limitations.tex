\subsection{Smart city learning}
The concept of smart-city has also been used in many different context and is associated with distinctive and innovative aspects that are often quite different. Big diversities are observed on the reasons \textit{why} different cities are defined as \textit{smart}.

This situation is the consequence of the lack of a clear and recognized definition of smart city.

Komninos\cite{komninos_intelligent_2002}, in his attempt to delineate the intelligent city, (perhaps the concept most closely related to the smart city), sees intelligent (smart) cities as ``\textit{territories with high capacity for learning and innovation, which is built\textendash in the creativity of their population, their institutions of knowledge creation, and their digital infrastructure for communication and knowledge management}''.

Smart cities are also a powerful ecosystem for learning. Smart city learning aim to support the improvement of all key factors contributing to the regional competitiveness: mobility, environment, people, quality of life and governance. The approach is aimed at optimizing resource consumption and saving time improving flows of people, goods and data\footnote{http://www.mifav.uniroma2.it/inevent/events/sclo/}.

Education in this context is pursued as a bottom-up process, where person and places are central. Smartness from a learning perspective exists both in the ambient data collected and among the communities that exists within a city.

The separation between student and teacher will fade out. Their role will be content or situation dependent: everybody will be a learner and the relation between persons will get a bigger role.


\subsection{Characteristics of Smart City Learning Applications}

%%...what are smart city learning applications?
%%...what kind of learning and what learning goals?

Technologies like mobile devices, tags, web based applications, geographical information and e-learning systems have already been used to develop smart city learning applications on the field\cite{perez-sanagustin_multichannel_2013}\cite{delfatto_geographic_2013}.

%%%%%%
Smart city learning applications consist in the implementation of urban informatics techniques and approaches to promote innovative engagement strategies\cite{amayocaldwell_urban_2013}.
Studies found that urban informatics provide an innovative opportunity to enrich students' place of learning within the city\cite{amayocaldwell_urban_2013}.

No doubt that among the consequences of such attention there is an acceleration in supporting the integration and embedding of ICT within physical environments to realize what has been defined the \textit{everyware}\cite{giffinger_smart_2007}.

% Other studies focused on the transformation from a \textit{space} of learning to a \textit{place} of learning in the smart city context, this is achieved through the exposure of real world urban sites and issues\cite{amayocaldwell_urban_2013}.

