Studies demonstrate that social connections in cities stimulates creativity and improves work quality \cite{florida_cities_2005}. This is only one of the reasons why the percentage of people living in urban environments is growing.

Smart cities present, by definition, a strong technological component.
In Technology Enhanced Learning (TEL), the role of technology is to direct, foster thinking and facilitate the acquisition of higher order skills \cite{goodyear_technologyenhanced_2010}.
Current research applied to learning in the cities seem to focus on two main technological means for learning contents: situated large displays and mobile devices, intended as tablets and smart-phones \cite{luff_mobility_1998}.

Traditional technology is a limiting factor: mobile devices and large screens support a very strict and confined set of interaction strategies. It's often not possible to tailor the user experience to properly fit the specific scenario because technology is too limiting.
Our goal is to design aiming at the best possible strategy for the users, building the technology around this process and avoiding the constraints typically introduced by more general-purpose hardware/software combinations.

We claim there is a space of opportunity for SCL applications in adopting novel ubiquitous computing approaches like tangible user interfaces (TUIs) and augmented objects (AOs). These technologies have already been found effective in supporting learning \cite{stanton_classroom_2001}, but their applications were mainly oriented to support learning as it happen in conventional schools and classrooms. The principal advantages in adopting these types of interfaces are (i) to enable the creation of rich and unobtrusive user experiences, (ii) to extend the type of data that can be captured to be used as learning content, including sensor data from the environment and from citizens' whereabouts. Therefore sensor-based TUIs could complement traditional approaches based on large screens and smartphones, especially when the learning environment can be as wide and heterogeneous as a city. Todays' increasingly adoption of sensors and IoT technologies are acting as enabling factors for the development of such interfaces; yet whether a number of studies have reported design guidelines for urban screens, there's a lack of guidelines to help the design of different types of interfaces.

As identified during a systematic mapping of the literature on smart city learning \cite{gianni_technologyenhanced_2016}, novel interaction modalities e.g. interactive objects and IoT, are not fully exploited. Even when used, the affordances employed are only a limited subset of the available ones. Unexplored opportunities emerged also when considering the learning aspect: rather than communities of citizens in the urban space, the research scenario usually involves schools or governance.

The need for more SCL research involving applications built around IoT and smart objects suggest the need to define a design space and a set of primitives, lying at different semantic levels, useful to structure and guide the authoring process.

%Toolkits play a role for facilitating the design and building prototypes of such applications.

% In this paper we...

%% ref toolkit


